This dissertation discusses various optimization techniques for the high-dimensional trajectory optimization of robots. We tackle two complex robot systems - multi-robot systems with a large number of robots planning start-to-goal collision-free trajectories and robot manipulators operating in $n$-dimensional joint spaces formed by $n$ joints. By suitable design of objective functions, choice of optimization paradigm, and mathematical approximations, our proposed path-planning methods aim to solve otherwise computationally intractable trajectory optimization problems. We present two major contributions - a GPU-accelerated distributed multi-agent trajectory optimizer and a stochastic trajectory optimizer for robot manipulators. 

For the GPU-accelerated multi-agent trajectory optimizer, we leverage parallelism offered by GPUs and mathematical reformulations to convert a complex Quadratic Programming(QP) optimization problem into a simple set of matrix-matrix products. We have outperformed existing state-of-the-art algorithms in various multi-robot planning scenarios with varied numbers of robots, obstacles, and their configurations. Beyond just computational acceleration, we also achieve comparable trajectory quality against our benchmarks and show how our optimizer achieves an improving performance gap with respect to them as the complexity of the planning task increases. 
\footnote{The software package for our GPU-accelerated distributed multi-robot optimizer has been made publicly available at \url{https://github.com/susiejojo/distributed_GPU_multiagent_trajopt}.}

For manipulator path planning, we adapt the existing Via-Point based Stochastic Trajectory Optimization(VPSTO) method with a suitably-designed cost function to plan collision-free paths in the joint space. We leverage PyBullet's mesh overlap-based collision detection to design our collision cost function. We demonstrate the efficiency of our approach through simulations in PyBullet for different manipulators such as UR5e and Franka Emika Panda. We then couple this high-level path planner with a low-level Deep Reinforcement Learning(RL) based controller to solve the non-prehensile task of pushing an object on a 2-D tabletop. We present results obtained from the bi-level trajectory optimizer for different shapes and numbers of obstacles on the table. 

Beyond computational speed-up, we demonstrate in Appendix \ref{sec:appendix-RVO} and \ref{sec:appendix-MAPF} a few additional applications of our multi-agent optimizer, in improving the quality of trajectories given by multi-agent collision-avoidance methods such as Reciprocal Velocity Obstacle(RVO) and Multi-agent Pathfinding(MAPF) Methods. This shows that our algorithm, when coupled with other planning algorithms as initializations, can improve the quality of trajectories planned by them. Beyond holonomic robots, our algorithm can also be extended to non-holonomic robot systems and even to high-dimensional robots such as manipulators. Similarly, our bi-level trajectory optimization algorithm for manipulators can be used as a teacher to train artificial neural networks to learn optimal paths from a given start position to an intended goal position. This approach could ultimately make the goal of real-time path planning for robot manipulators a reality. 

In the entirety, we believe that the contributions of this dissertation will help solve path-planning problems for a wide variety of complex robot systems and will serve as inspiration to enable robots to solve tasks ranging from simplifying our daily household work to industrial labor and defense applications. 
